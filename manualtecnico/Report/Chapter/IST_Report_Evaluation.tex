%%%%%%%%%%%%%%%%%%%%%%%%%%%%%%%%%%%%%%%%%%%%%%%%%%%%%%%%%%%%%%%%%%%%%%%%
%                                                                                            %
%     File: IST_Report_Evaluation.tex                                    %
%                                                                                             %
%%%%%%%%%%%%%%%%%%%%%%%%%%%%%%%%%%%%%%%%%%%%%%%%%%%%%%%%%%%%%%%%%%%%%%%%
% !TEX root = IST_Report.tex
\newpage
\chapter{Avaliação}
\label{chapter:evaluation}
De modo a confirmar a segurança do nosso sistema, adulteramos as mensagens e 
assinaturas realizadas. Confirmarmos que o sistema detecta corretamente a adulteração de mensagens. 
\\
Foram realizados vários testes de desempenho do sistema utilizando um computador de utilizador comum MacBook Pro com processador 2.9 GHz Intel Core i7, memória 8 GB 1600 MHz DDR3 e sistema operativo MacOsX 10.8.3 e disco SSD (evidênciado na latência de escrita em disco). O servidor de timestamp foi executado localmente pelo que não são considerados atrasos na rede.\\
Foram realizadas 5 amostras por cada caso e determinados os níveis de confiança a 95\% para os tempos obtidos. O directório enviado foi constituído por  286 ficheiros PDF e Word num total de 100MBytes. \\
\begin{itemize}
  \item \textbf{Compressão: }A compressão do ficheiro converte $100MB$ em $95.5MB$. Esta taxa depende do formato e contéudo dos ficheiros de origem. Demorou em média $4484ms$ com desvio padrão de $30ms$. 
  \item \textbf{Hashing do ficheiro: } A realização do hash do ficheiro comprimido para utilizar nas assinaturas demora em média $700ms$ com desvio padrão de $10ms$.
  \item \textbf{Assinatura com cartão de cidadão:} A assinatura com cartão de cidadão demorou $936ms$ com desvio padrão de $24ms$. 
 \item \textbf{Timestamp: }O serviço de timestamp demora $70ms$ a decorrer com desvio padrão de $5ms$.
\item \textbf{Codificação e escrita em disco: }A conversão para base 64 e escrita em disco demorou $1335ms$ com desvio padrão de $35ms$. A codificação de base 64 implica um aumento da dimensão do ficheiro ao factor de $8/6$ pelo que o ficheiro final tem $130.7MB$.
\item  \textbf{Descompressão e escrita em disco}A descompressão demorou $1380ms$ com desvio padrão $32ms$.
\item \textbf{Verificação timestamp: }A verificação do timestamp demorou $32ms$ com desvio padrão de $6ms$.
\item  \textbf{Verificação Assinatura: } A verificação da Assinatura demorou $6ms$ com desvio padrão de $3ms$.
\item \textbf{Cifra e Decifra: } A especificação técnica da caixa fornecida pelo 
cliente indica $60 segundos$ para cada $100MB$, pelo que assumimos este valor na 
ausência da possibilidade de realização de testes experimentais.
\end{itemize}
No total, o processo de envio demora (excluindo a cifra) \textbf{6 segundos} enquanto que 
o processo no receptor demora \textbf{3 segundos}. Estes valores não são significativos para a 
experiência do utilizador quando comparados com o tempo de inserção do PIN do 
cartão de cidadão e da cifra pela caixa fornecida pelo cliente.\\
A excelente performance do nosso sistema deve-se em parte ao facto de os dados só serem lidos e escritos uma única vez no disco. 
O reduzido numero de acessos a disco reduz o atraso do processo. Por outro lado implica a utilização de mais memória RAM do utilizador. Este processo teve um pico de utilização máxima de memória de 300MB. Consideramos que este custo não é significativo dadas as especificações atuais dos computadores para utilizador final. Caso o utilizador tenha menos memória, a solução é aplicável mas  para ficheiros de menores dimensões.\\
A análise de performance conclui portanto que a nossa solução é eficiente.\\

