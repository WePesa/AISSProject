%%%%%%%%%%%%%%%%%%%%%%%%%%%%%%%%%%%%%%%%%%%%%%%%%%%%%%%%%%%%%%%%%%%%%%%%
%                                                                      %
%     File: IST_Report_Introduction.tex                                    %
%                                                                      %
%%%%%%%%%%%%%%%%%%%%%%%%%%%%%%%%%%%%%%%%%%%%%%%%%%%%%%%%%%%%%%%%%%%%%%%%
% !TEX root = IST_Report.tex
\chapter{Introdução}
\label{chapter:introduction}
Este documento descreve a implementação de um Serviço de Emails Seguro. Esta implementação teve como premissa o pedido de um \textbf{cliente} para criar um plugin para o programa de emails \textit{Mozilla Thunderbird}. \\
Este plugin teria que ter a possibilidade de garantir a \textbf{confidencialidade}, \textbf{não repudiação}, \textbf{autenticidade} e, como resultado de um requisito opcional por parte do Cliente, \textbf{garantia 
temporal} de uma mensagem trocada entre 2 utilizadores. Além disso, a garantia de 
\textit{confidencialidade} teria de ser realizada através da Cifra/Decifra AES por uma caixa fornecida pelo 
Cliente e a \textit{não repudiação} teria de ser assegurada através da assinatura da 
mensagem com o Cartão de Cidadão da República Portuguesa. O Cliente não exigiu quaisquer requísito extra em relação à \textit{garantia temporal}, apenas que deve atribuir e validar um Timestamp Seguro à mensagem, sendo tal como as restantes operações: Cifra e Assinatura, de caracter opcional aquando da realização da operação.
\\
Desta forma identificámos, no serviço a disponibilizar,  2 sujeitos principais:
\begin{itemize}
\item Sender -  Tem como input uma pasta de dados que pode comprimir, cifrar, assinar e adicionar timestamp 
e gera um output para ser enviado ao destinatário (por exemplo por email)
\item Receiver - Tem como input o email seguro do sender que irá decifrar, verificar assinatura, timestamp 
e descomprimir para receber o conteúdo original.
\end{itemize}
A solução apresentada cumpre os requisitos de segurança estabelecidos pelo 
Cliente: utilizando uma interface própria que gera um ficheiro que pode não só 
ser enviado pelo cliente de email  \textit{Thunderbird} como por qualquer outro 
sistema de transferência de texto ou dados.\\

No \textbf{Capítulo~\ref{chapter:architecture}} é apresentada  a Arquitectura do Sistema constituido pelo módulo cliente (Sender ou Receiver) onde são implementadas as funcionalidades de cifra, assinatura e compressão, e pelo servidor de timestamp seguro.\\
No \textbf{Capítulo~\ref{chapter:evaluation}} é feita uma Avaliação à performance e desempenho do sistema, tendo em conta as várias operações e a sua contextualização em cenário real. \\
No \textbf{Capítulo~\ref{chapter:conclusions}} são tiradas as conclusões sobre o trabalho efectuado e respectivas reflexões.
\newpage




