%%%%%%%%%%%%%%%%%%%%%%%%%%%%%%%%%%%%%%%%%%%%%%%%%%%%%%%%%%%%%%%%%%%%%%%%
%                                                                      %
%     File: IST_Report_Introduction.tex                                    %
%                                                                      %
%%%%%%%%%%%%%%%%%%%%%%%%%%%%%%%%%%%%%%%%%%%%%%%%%%%%%%%%%%%%%%%%%%%%%%%%
% !TEX root = IST_Report.tex
\chapter{Introdução}
\label{chapter:introduction}
Este documento descreve a implementação de um Serviço de Emails Seguro. Esta implementação teve como premissa o pedido de um \textbf{cliente} de uma proposta de plugin que se adaptasse, preferencialmente, ao cliente de emails \textit{Mozilla Thunderbird} e que efectuasse as seguintes operações de segurança:
\begin{itemize}
\item Cifra/Decifra AES com chave gerada por caixa fornecida pelo Cliente
\item Assinatura com Cartão de Cidadão da República Portuguesa
\item Assinatura e Verificação de Timestamp Seguro
\end{itemize}

Desta forma o Serviço de Emails Seguro tem dois sujeitos:
\begin{itemize}
\item Sender -  Tem como input uma pasta de dados que pode comprimir, cifrar, assinar e adicionar timestamp 
e gera um output para ser enviado ao destinatário (por exemplo por email)
\item Receiver - Tem como input o email seguro do sender que irá decifrar, verificar assinatura e timestamp 
e descomprimir para receber o conteúdo original.
\end{itemize}
No \textbf{Capítulo~\ref{chapter:architecture}} é apresentada  a Arquitectura do Sistema constituido pelo módulo cliente onde são implementadas as funcionalidades de cifra, assinatura e compressão, e pelo servidor de timestamp seguro.\\
No \textbf{Capítulo~\ref{chapter:conclusions}} são tiradas as conclusões sobre o trabalho efectuado.
\newpage




