%%%%%%%%%%%%%%%%%%%%%%%%%%%%%%%%%%%%%%%%%%%%%%%%%%%%%%%%%%%%%%%%%%%%%%%%
%                                                                      %
%     File: IST_Report_Introduction.tex                                    %
%                                                                      %
%%%%%%%%%%%%%%%%%%%%%%%%%%%%%%%%%%%%%%%%%%%%%%%%%%%%%%%%%%%%%%%%%%%%%%%%
% !TEX root = IST_Report.tex
\chapter{Introdução}
\label{chapter:introduction}


Este manual descreve a nossa implementação de um Serviço de Emails Seguro. Esta implementação teve como premissa o pedido de um \textbf{cliente} de uma proposta de plugin que se adaptasse, preferencialmente, ao cliente de emails \textit{Mozilla Thunderbird} e que efectuasse as seguintes operações de segurança:
\begin{itemize}
\item Cifra/Decifra AES com chave gerada por caixa fornecida pelo Cliente
\item Assinatura com Cartão de Cidadão da República Portuguesa
\item Assinatura e Verificação de Timestamp Seguro
\end{itemize}

Desta forma o Serviço de Emails Seguro, tal como qualquer cliente de email tem dois principais sujeitos:
\begin{itemize}
\item Sender - quem envia o email e pode cifrar, assinar e adicionar timestamp a este
\item Receiver - quem recebe o email seguro e pode decifrar, verificar assinatura e timestamp
\end{itemize}

A arquitectura e algoritmo foram pensados não só tendo em conta as exigências do cliente - seja a utilização de uma caixa de cifra ou do Cartão de Cidadão e existênca de portabilidade - mas, também, as melhores práticas em implementação de algoritmos de Segurança.

No \textbf{Capítulo~\ref{chapter:architecture}} é apresentada toda a Arquitectura do Sistema constituída por um Cliente que ora pode receber ora enviar emails para outro Cliente e por um Servidor de Timestamps, iinterligação entre estes módulos e uma explicação mais detalhada do funcionamento dos vários procedimentos como, por exemplo, cifra e assinatura.

No \textbf{Capítulo~\ref{chapter:conclusions}} são tiradas algumas conclusões sobre o trabalho efectuado.




%Deve terminar com um sumário do resto do relatório. Por exemplo: ``Na Secção~\ref{section:recomenda} %descreve-se ..., e no Capítulo~\ref{chapter:conclusions} apresentam-se as conclusões.''


%\section{Recomendações}
%\label{section:recomenda}

%Faça citações usando o seguinte modo: \cite{delgado:2010,arroz:2007}. Se precisar de ter Tabelas use este %exemplo:

%\begin{table}[h!]
%  \begin{center}
%    \begin{tabular}{|c|c|}
 %     \hline
  %    item 1 & item 2 \\
   %   \hline
    %  item 3 & item 4 \\
     % \hline
    %\end{tabular}
   % \caption[A legenda da Tabela aparece indexada]{Legenda da Tabela}
   % \label{tabela:simple}
 % \end{center}
% \end{table}

% Referências a Tabelas são feitas do seguinte modo: Tabela~\ref{tabela:simple}.


