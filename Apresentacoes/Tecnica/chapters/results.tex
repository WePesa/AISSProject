
%====================================================================================	 
\section{Results}
\begin{frame}{Results}

\begin{block}{Simulation}
\begin{itemize}
\item MacBook Pro 2.9 GHz Intel Core i7
\item 8 GB memory 1600 MHz DDR3, SSD disk
\item Mac OS X 10.8.3
\end{itemize}
\end{block}	

\begin{block}{Performance metrics}
\begin{itemize}
\item \textit{286 documents (PDF and Word)}
\item \textit{Total Size: 100 MBytes}
\end{itemize}
\end{block}	
\end{frame}

%\begin{table}[H]
%	\centering
%	\includegraphics[width=1.01\textwidth]{images/table1.pdf}	
 %	\caption{The benchmark applications used in experiments.}
%	\label{fig:}
%\end{table}
	
\end{frame}

%====================================================================================
\begin{frame}{Results \\ \small Trace-driven simulation experiments}


\begin{itemize}
\item \textbf{Compression: } From $100MB$ to $95.5MB$. This value depends on the content and format of original files. Average Time: $4484ms$, Standard Deviation: $30ms$. 
\item \textbf{Hashing: } Average Time: $700ms$,  Standard Deviation: $10ms$.
\item \textbf{Signature with Citizen Card:} Average Time: $936ms$, Standard Deviation: $24ms$. 
\item \textbf{Timestamp: } Average Time: $70ms$, Standard Deviation: $5ms$.
\item \textbf{To Base64 and persistent write: }Average Time: $1335ms$, Standard Deviation: $35ms$. With Base64 file proportion increases $8/6$ - final size: $130.7MB$.
\item  \textbf{From Base64}Average Time: $1380ms$, Standard Deviation: $32ms$.
\item \textbf{Timestamp Verification: } Average Time: $32ms$, Standard Deviation: $6ms$.
\item  \textbf{Signature Verification: } Average Time: $6ms$, Standard Deviation: $3ms$.
\item \textbf{Cipher/Decipher: } Technical specification of box is: $60 seconds$ for each $100MB$, 
\end{itemize}

Total time for sending process (excluding cipher) \textbf{6 seconds}\\
Total time for reception process: \textbf{3 seconds}. \\

%\begin{figure}[H]
%	\centering
%	\includegraphics[width=0.9\textwidth]{images/fig4.pdf}	
% 	\caption{Comparison of total offloading delay for four different decision-making algorithms.}
%	\label{fig:}
%\end{figure}

\end{frame}

%==============================================================================
\begin{frame}{Results \\ \small Prototype experiments}
\begin{block}{Mobile Device (emulated)}
\begin{itemize}
\item 266 MHz HP laptop;
\item 11 Mbps 802.11b
\end{itemize}
\end{block}	

\begin{block}{Surrogate}
\begin{itemize}
\item 733 MHz HP Kayak PC workstation
\item 128 Mbytes of RAM
\end{itemize}
\end{block}	

\begin{block}{Network}
\begin{itemize}
\item 10Mbps
\end{itemize}
\end{block}	


\end{frame}

\begin{frame}{Results \\ \small Prototype experiments}


%\begin{figure}[H]
%	\centering
%	\includegraphics[width=0.8\textwidth]{images/fig5.pdf}	
%	\caption{Execution time of two applications under the offloading prototype. The bars show the time until the %executions exited. The X’s indicate that the applications failed before completion.}
%	\label{fig:}
%\end{figure}


\end{frame}

\begin{frame}{Results \\ \small Prototype experiments}


\begin{block}{Offloading, constrained memory}
\begin{itemize}
\item 1.5\% - 5.7\% more slow;
\item Cost of monitoring, remote accesses and partitioning increases the execution time.
\end{itemize}
\end{block}	
\end{frame}

